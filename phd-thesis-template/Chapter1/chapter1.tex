%*******************************************************************************
%*********************************** First Chapter *****************************
%*******************************************************************************

\chapter{Introdcution}  %Title of the First Chapter

\ifpdf
    \graphicspath{{Chapter1/Figs/Raster/}{Chapter1/Figs/PDF/}{Chapter1/Figs/}}
\else
    \graphicspath{{Chapter1/Figs/Vector/}{Chapter1/Figs/}}
\fi


%********************************** %First Section  **************************************
\section{Motivation}
With the advent of medical imaging, computer aided diagnostic systems have become an integral part of today's medical diagnosis\cite{doi2007computer}. CAD systems have become a part of routine clinical work and are being used extensively for disesase diagnosis.A myriad of different medical imaging systems,like X-Ray, Magnetic Resonance Imaging (MRI), Computed Tomography Scans etc, are used for diagnosis. The output of such systems are multi dimensional digital images, interpretation of which require sophisticated digital image processing methods. Automated medical diagnosis systems can aid in easy and faster interpretation of these images.\\	

Digital fundus imaging in ophthlamology is a vital component in diagnosis of various pathologies. Retinal vessel segmentation forms an important part of diagnosis of such pathologies. Changes in the retinal vascualture is precursor to many diseases such as diabetes,hypertension and stroke. Morphological properties such as diameter, length, branching angle, of the retinal vessel forms an important component in diagnosis and evaluation of opthalmologic diseases such as diabetic retinopathy \cite{sinthanayothin2002automated} and hyperternsion. For example, vessel diameter measurement can be an aid in diagnosis of hypertension\cite{calvo2011automatic}.\\

Vessel segmentation in itself a challenging and a tedious task which may take a couple of hours one done manually. Low contrast between vessel and background, noise in the image and variability in the width, brightness and shape alongwith the presence of exudates,lesions, hemorrhage spots and other pathological effects make the task much more difficult. Figure \ref{fig:fundusdiseased} shows a fundus image of a diseased eye.\\

\begin{figure}
		\centering	
		\includegraphics[width=0.5\textwidth]{fundusimage.png}
		\caption{A diseased fundus image}
		\label{fig:fundusdiseased}		
\end{figure}	

Developing an automated retinal vessel segmentation is a first step towards developing a full fledged CAD system for diagnosis of opthalmology pathologies. There have been a lot of work in literature on automatic retinal segmentation, including based on matched filtering, tracking methods, morphological methods and learning based methods. Some of the simplest approaches to segmentation use adaptive thresholding to segment out the blood vessels. Another simple approach is to use edge detection tecniques like canny edge detector for vessel segmentation. Some of the more sophsticated models use learning based algorithms which learn on a set of given image segmentations. Many of these models, learn local features at a pixel location and train a pixel based classifier.\citep{nguyen2011effective} present the limitations in some of the state-of-art methods and present a multi scale line detection based method for blood vessel segmentation.Issues, like poor segmentation at bifurcation and croosover regions, merging of close vessels and poor segmentation of small vessels limit the application of vasculature based medical diagnosis.For example, merging of two nearby vessels can lead to vessel being considered as one wide vessel, thereby affecting the width measurementes of the vessel. These limitations may contribute to inaccuracies in vascular network analysis and subsequently in characterization of diseases. \\

Supervised learning methods, in general are limited by the amount of training data.Also most of them are restricted to the type of training data and do not generalise on the task at hand. In our thesis, we propose two supervised learning models for accurate vessel segmentation. Our method is a patch-based framework and learns the local structure of the vessel at patch level. The proposed model, exploits the presence of common stuctures in a typical vasculature tree. Unlike most of the supervised learning methods which make a prediction at pixel level, classifying each pixel as vessel or background, our model predicts the local structure around each pixel at patch level. Figure, shows the basic architecture of our method.

\section{Outline of thesis}
The thesis is organized as follow,
